\documentclass{article}

\usepackage[final]{neurips_2022}
\usepackage[UTF8]{ctex}

\usepackage[utf8]{inputenc}
\usepackage[T1]{fontenc}
\usepackage{hyperref}
\usepackage{url}
\usepackage{booktabs}
\usepackage{amsfonts}
\usepackage{nicefrac}
\usepackage{float}
\usepackage{amsmath}
\usepackage{algorithm}
\usepackage{algpseudocode}
\usepackage{microtype}
\usepackage{xcolor}
\usepackage{bm}
\usepackage{graphicx}
\usepackage{subfigure}
\usepackage{listings}

\hypersetup{
    colorlinks=true,
    linkbordercolor=white,
}

\renewcommand{\algorithmicrequire}{\textbf{Input:}}
\renewcommand{\algorithmicensure}{\textbf{Output:}}

\title{大数据分析中的算法期末上机报告}

\author{%
    \large 陈润璘 \\
    \large \texttt{2200010848}
    \And
    \large 任子博 \\
    \large \texttt{2200010626}
}

\begin{document}

    \maketitle


    \section{问题描述}

    \subsection{时空网络模型}\label{subsec:spatial-temporal-network-model}

    铁路列车时刻表问题的时空网络模型旨在通过一个0-1整数规划模型来确定一个无冲突的列车运行计划。该模型基于时空网络构建,目标是最大化某种意义上的“利润”。

    \subsubsection{目标函数}
    模型的目标是最大化所有调度列车在各自路径上所选弧(路段或车站停驻)的“利润”之和:
    \begin{equation}
        \max \sum_{j \in J} \sum_{e \in E_j} p_e x_e\label{eq:obj}
    \end{equation}
    其中,$p_e$ 是使用弧 $e$
    的“利润”值。在实现中,“利润”可以有多种实际含义,例如:若将始发弧的利润设为1,其他为0,则目标是最大化运行的列车数量;若将$p_e$设为路段运行时间的相反数,则目标是最小化总运行时间;若将目标函数直接设为0,则仅进行可行性检查。

    \subsubsection{决策变量}
    \begin{itemize}
        \item $x_e \in \{0,1\}$: 二元决策变量。如果弧 $e \in E$ 被某列车占用,则 $x_e = 1$;否则为 $0$。
        \item $z_{jv}$: 辅助二元变量。如果节点 $v \in V_j$ 被列车 $j \in J$ 占用,则
        $z_{jv} = 1$;否则为 $0$。
        \item $y_v$: 辅助二元变量。如果节点 $v \in V_j$ 被任何列车占用,则 $y_v = 1$;否则为 $0$。
    \end{itemize}

    \subsubsection{模型参数}
    \begin{itemize}
        \item $J$: 所有列车的集合 。
        \item $E_j$: 列车 $j$ 在时空网络中可使用的弧(arc)的集合。
        \item $E$: 时空网络中所有弧的集合, $E = \bigcup_{j \in J} E_j$ 。
        \item $V_j$: 列车 $j$ 在时空网络中可访问的节点(node)的集合 。
        \item $V$: 时空网络中所有节点的集合, $V = \bigcup_{j \in J} V_j$ 。
        \item $p_e$: 列车使用弧 $e$ 所产生的“利润” 。
        \item $\sigma, \tau$: 分别表示时空网络中的虚拟始发节点和虚拟终点节点 。
        \item $\delta_j^+(v)$: 对于列车 $j$,从节点 $v$ 出发的弧的集合 。
        \item $\delta_j^-(v)$: 对于列车 $j$,进入节点 $v$ 的弧的集合 。
        \item $T(v)$: 可能通过节点 $v$ 的所有列车的集合 。
        \item $N(v)$: 与节点 $v$ 相冲突的节点集合(包括 $v$ 本身)。这些节点不能同时被占用 。
    \end{itemize}

    \subsubsection{约束条件}
    \begin{enumerate}
        \item \textbf{列车路径的始发约束}:
        每列车 $j$ 最多只能选择一条从虚拟始发节点 $\sigma$ 出发的弧。这表示每列车最多只能开始其行程一次。
        \begin{equation}
            \sum_{e \in \delta_j^+(\sigma)} x_e \le 1, \quad \forall
            j \in J\label{eq:con_start}
        \end{equation}

        \item \textbf{流量守恒约束}:
        对于每列车 $j$ 和每个非虚拟始发/终点节点
        $v$,进入该节点的被选中的弧的数量必须等于从该节点出发的被选中的弧的数量。这确保了列车路径的连续性。
        \begin{equation}
            \sum_{e \in \delta_j^-(v)} x_e = \sum_{e \in
            \delta_j^+(v)} x_e, \quad \forall j \in J, \forall v \in
            V \setminus \{\sigma, \tau\}\label{eq:con_flow}
        \end{equation}

        \item \textbf{列车路径的终止约束}:
        每列车 $j$ 最多只能选择一条进入虚拟终点节点 $\tau$ 的弧。这表示每列车最多只能结束其行程一次 。
        \begin{equation}
            \sum_{e \in \delta_j^-(\tau)} x_e \le 1, \quad \forall j
            \in J\label{eq:con_end}
        \end{equation}

        \item \textbf{节点占用逻辑约束}:
        定义了节点 $v$ 是否被列车 $j$ 占用 ($z_{jv}$)。如果列车 $j$ 使用了任何一条进入节点 $v$
        的弧,则该节点被列车 $j$ 占用。
        \begin{equation}
            z_{jv} = \sum_{e \in \delta_j^-(v)} x_e, \quad \forall j
            \in J, \forall v \in V_j\label{eq:con_node_occupied}
        \end{equation}

        \item \textbf{节点总体占用逻辑约束}:
        定义了节点 $v$ 是否被任何列车占用 ($y_v$)。如果至少有一列车占用了节点 $v$,则 $y_v=1$。
        \begin{equation}
            y_v = \sum_{j \in T(v)} z_{jv}, \quad \forall v \in
            V_j\label{eq:con_node_occupied_total}
        \end{equation}

        \item \textbf{Headway (最小间隔) 约束}:
        对于网络中的任何节点 $v$,其冲突集合 $N(v)$
        中的所有节点(这些节点代表了在时间或空间上不能同时占用的状态),最多只能有一个被占用。这是为了避免列车冲突,保证运行安全。
        \begin{equation}
            \sum_{v' \in N(v)} y_{v'} \le 1, \quad \forall v \in
            V\label{eq:con_headway}
        \end{equation}

        \item \textbf{二元变量约束}:
        所有决策变量 $x_e$ 都必须是0或1。
        \begin{equation}
            x_e \in \{0,1\}, \quad \forall e \in E\label{eq:con_binary}
        \end{equation}
    \end{enumerate}

    \subsection{Job-Shop调度模型}

    \subsubsection{目标函数}
    Job-Shop 调度模型支持三种目标函数:

    \begin{enumerate}
        \item \textbf{最小化所有列车到达终点站的总时间}:
        \begin{equation}
            \min \sum_{j \in J} a_{j, S}\label{eq:min\_sum\_arrival}
        \end{equation}
        其中 $a_{j, S}$ 表示列车 $j$ 到达终点站 $S$ 的时间。

        \item \textbf{最小化最大完工时间(Makespan)}:
        \begin{equation}
            \min C_{\max} , \ \ text{其中} \ C_{\max} = \max_{j \in J} a_{j, S}\label{eq:min}
        \end{equation}

        \item \textbf{可行性检查}(feasibility):
        \begin{equation}
            \min 0\label{eq:feasibility}
        \end{equation}
        仅判断约束是否可行,不优化具体目标。
    \end{enumerate}

    \subsubsection{决策变量}
    \begin{itemize}
        \item $a_{j,s}$: 连续变量,表示列车 $j \in J$ 到达车站 $s \in \{1, \dots, N_S-1\}$ 的时间。
        \item $d_{j,s}$: 连续变量,表示列车 $j \in J$ 从车站 $s \in \{0, \dots, N_S-2\}$ 出发的时间。
        \item $\delta^{arr}_{j_1,j_2,s}$: 二元变量,用于处理列车 $j_1, j_2 \in J$ ($j_1 \neq j_2$) 在车站 $s \in \{1, \dots, N_S-1\}$ 的到达间隔。如果 $j_1$ 在 $j_2$ 之前(或同时)到达,则可能为1,否则为0(具体取决于约束的精确形式)。
        \item $\delta^{dep}_{j_1,j_2,s}$: 二元变量,用于处理列车 $j_1, j_2 \in J$ ($j_1 \neq j_2$) 在车站 $s \in \{0, \dots, N_S-2\}$ 的出发间隔。如果 $j_1$ 在 $j_2$ 之前(或同时)出发,则可能为1,否则为0。
        \item $\pi_{j_1,j_2,s}$: 二元变量,表示列车 $j_1, j_2 \in J$ ($j_1 \neq j_2$) 在始发站为 $s \in \{0, \dots, N_S-2\}$ 的运行区段 $(s, s+1)$ 上的运行顺序。如果 $j_1$ 先于(或同时于) $j_2$ 通过该区段,则为1,否则为0。
        \item $C_{max}$: 连续变量,表示所有列车的最大完工时间(即最晚到达终点站的时间)。仅在目标函数为 \texttt{min\_makespan} 时使用。
    \end{itemize}

    \subsubsection{模型参数}
    \begin{itemize}
        \item $J$: 所有列车的集合。
        \item $N_S$: 车站的总数量。车站索引从 $0$ (始发站) 到 $N_S-1$ (终点站)。
        \item $R_{j,s,s+1}$: 列车 $j$ 在区段 $(s, s+1)$ 的纯运行时间。
        \item $D_{min\_stop}$: 列车在停靠站的最小停站时间。
        \item $D_{max\_stop}$: 列车在停靠站的最大停站时间。
        \item $D_{pass}$: 列车通过不停站时的名义停站时间(通常为0)。
        \item $H$: 最小安全行车间隔时间 (Headway)。
        \item $T_{max}$: 列车运行图的最大允许时间范围。
        \item $M$: 一个足够大的正数,用于线性化逻辑约束。
        \item $\epsilon$: 一个很小的正数,用于在约束中严格不等关系。
        \item $is\_stop_{j,s}$: 一个指示参数,如果列车 $j$ 计划在车站 $s$ 停站,则为真(或1),否则为假(或0)。
    \end{itemize}

    \subsubsection{约束条件}
    \begin{enumerate}
        \item \textbf{运行时间约束 (Travel Time Constraints)}: 对于每列车 $j \in J$ 和每个运行区段 $(s, s+1)$ (其中 $s \in \{0, \dots, N_S-2\}$):
        \begin{equation}
            a_{j,s+1} = d_{j,s} + R_{j,s,s+1}\label{eq:travel_time}
        \end{equation}

        \item \textbf{停站时间约束 (Dwell Time Constraints)}: 对于每列车 $j \in J$ 和每个中间车站 $s \in \{1, \dots, N_S-2\}$:
        \begin{itemize}
            \item 如果列车 $j$ 在车站 $s$ 停站 ($is\_stop_{j,s} = \text{true}$):
            \begin{equation}
                \begin{align}
                    d_{j,s} &\ge a_{j,s} + D_{min\_stop} \\
                    d_{j,s} &\le a_{j,s} + D_{max\_stop}
                \end{align}\label{eq:dwell_time}
            \end{equation}
            \item 如果列车 $j$ 在车站 $s$ 通过 ($is\_stop_{j,s} = \text{false}$):
            \begin{equation}
                d_{j,s} = a_{j,s} + D_{pass}\label{eq:dwell_time2}
            \end{equation}
        \end{itemize}

        \item \textbf{到达间隔约束 (Arrival Headway Constraints)}: 对于任意两列不同的列车 $j_1, j_2 \in J$ ($j_1 \neq j_2$) 和每个车站 $s \in \{1, \dots, N_S-1\}$ (非始发站):
        \begin{equation}
            \begin{align}
                a_{j_2,s} &\ge a_{j_1,s} + H - M \cdot (1 - \delta^{arr}_{j_1,j_2,s}) \\
                a_{j_1,s} &\ge a_{j_2,s} + H - M \cdot \delta^{arr}_{j_1,j_2,s}
            \end{align}\label{eq:arrival_headway}
        \end{equation}

        \item \textbf{出发间隔约束 (Departure Headway Constraints)}: 对于任意两列不同的列车 $j_1, j_2 \in J$ ($j_1 \neq j_2$) 和每个车站 $s \in \{0, \dots, N_S-2\}$ (非终点站):
        \begin{equation}
            \begin{align}
                d_{j_2,s} &\ge d_{j_1,s} + H - M \cdot (1 - \delta^{dep}_{j_1,j_2,s}) \\
                d_{j_1,s} &\ge d_{j_2,s} + H - M \cdot \delta^{dep}_{j_1,j_2,s}
            \end{align}
            \label{eq:departure_headway}
        \end{equation}

        \item \textbf{区间占用顺序约束 (Block Section Order - FIFO)}: 对于任意两列不同的列车 $j_1, j_2 \in J$ ($j_1 \neq j_2$) 和每个运行区段的始发车站 $s \in \{0, \dots, N_S-2\}$ (该区段为 $s \to s+1$):
        \begin{equation}
            \begin{align}
                d_{j_1,s} &\le d_{j_2,s} + M \cdot (1 - \pi_{j_1,j_2,s}) \\
                a_{j_1,s+1} &\le a_{j_2,s+1} + M \cdot (1 - \pi_{j_1,j_2,s}) \\
                d_{j_2,s} &\le d_{j_1,s} - \epsilon + M \cdot \pi_{j_1,j_2,s} \\
                a_{j_2,s+1} &\le a_{j_1,s+1} + M \cdot \pi_{j_1,j_2,s}
            \end{align}
            \label{eq:block_section_order}
        \end{equation}
        这些约束确保如果 $\pi_{j_1,j_2,s}=1$,则列车 $j_1$ 在区段 $(s, s+1)$ 上先于或同时于 $j_2$ 出发和到达;如果 $\pi_{j_1,j_2,s}=0$,则 $j_2$ 严格先于 $j_1$ 从车站 $s$ 出发,并先于或同时于 $j_1$ 到达车站 $s+1$。

        \item \textbf{时间窗约束 (Time Window Constraints)}: 对于每列车 $j \in J$:
        \begin{itemize}
            \item 列车到达终点站 $N_S-1$ 的时间不晚于 $T_{max}$:
            \begin{equation}
                a_{j,N_S-1} \le T_{max}\label{eq:arrival_time_max}
            \end{equation}
            \item 列车从所有非终点站 $s \in \{0, \dots, N_S-2\}$ 出发的时间不晚于 $T_{max}$:
            \begin{equation}
                d_{j,s} \le T_{max}\label{eq:departure_time_max}
            \end{equation}
        \end{itemize}

        \item \textbf{变量定义域约束 (Variable Domain Constraints)}:
        \begin{itemize}
            \item 所有到达和出发时间为非负连续变量: $a_{j,s} \ge 0$, $d_{j,s} \ge 0$。
            \item 所有 $\delta^{arr}_{j_1,j_2,s}$, $\delta^{dep}_{j_1,j_2,s}$, $\pi_{j_1,j_2,s}$ 变量为二元变量, 即 $\in \{0,1\}$。
            \item 如果使用 $C_{max}$,则 $C_{max} \ge 0$。
        \end{itemize}

        \item \textbf{Makespan 约束} (仅当目标为最小化最大完工时间时): 对于每列车 $j \in J$:
        \begin{equation}
            C_{max} \ge a_{j,N_S-1}\label{eq:makespan_constraint}
        \end{equation}
    \end{enumerate}


    \section{算法描述}
    \label{sec:algo_desc}

    \subsection{拉格朗日松弛算法 (Lagrangian Relaxation Solver)}
    拉格朗日松弛算法是一种用于求解整数规划问题的近似方法,特别适用于具有复杂约束的调度问题。该方法通过将某些约束引入目标函数中,并使用拉格朗日乘子来调整这些约束的影响,从而简化问题求解。
    本次作业基于拉格朗日松弛框架,主要针对时空网络模型中的 headway 约束(式\ref{eq:con_headway}) 进行松弛,算法伪代码如\ref{alg:lagrangian_relaxation}所示,以下为具体实现细节:

    \begin{algorithm}[h]
        \caption{Lagrangian Relaxation Solver}
        \begin{algorithmic}[1]
            \Require 问题数据 (时空网络, 列车信息, headway 定义)
            \Ensure 近似最优且可行的列车路径集合 $X^*$
            \State 初始化拉格朗日乘子 $\lambda_q \ge 0$ (针对 headway 约束 $q$)
            \State 初始化步长 $\eta > 0$, 上下界 $LB, UB$, 最优可行解 $X^*$
            \For{$k = 1 \dots K_{max}$ (最大迭代次数)}
                \ForAll{列车 $j \in J$}
                    \State 根据当前 $\lambda$ 计算修正后的弧利润 $p'_e = p_e -
                    \text{penalty}(\lambda, e)$
                    \State 求解列车 $j$ 的最长路径问题得到路径 $X^k_j$ 及其对偶利润
                \EndFor
                \State 计算当前拉格朗日对偶函数值 $L_k(\lambda)$ (所有列车对偶利润之和 + $\sum \lambda_q$)
                \State 更新上界 $UB = \min(UB, L_k(\lambda))$
                \State 使用启发式方法 (如基于子问题解排序,逐个调度列车并避免冲突) 得到可行路径集 $X_{feas}$
                和目标值 $Z_{feas}$
                \If{$Z_{feas} > LB$}
                    \State 更新下界 $LB = Z_{feas}$
                    \State $X^* = X_{feas}$
                \EndIf
                \If{满足终止条件 (如 $UB-LB$ 足够小且 $X^*$ 可行)}
                    \State \textbf{break}
                \EndIf
                \State 计算 headway 约束的违反度 $g_q$
                \State $\lambda_q \gets \max(0, \lambda_q + \eta \cdot g_q)$
                \State 更新步长 $\eta$
            \EndFor
            \State \Return 最优可行解 $X^*$
        \end{algorithmic}\label{alg:lagrangian_relaxation}
    \end{algorithm}

    \begin{itemize}
        \item \textbf{松弛问题}: Headway 约束 $\sum_{v' \in N(v)} y_{v'} \le 1$
        被引入目标函数中,带有拉格朗日乘子 $\lambda_q$
        (代码中为 \texttt{lambdas},其中 $q$ 代表一个 headway
        约束窗口,由节点类型、站点索引和锚定时间定义)。松弛后的目标函数变为 $\max
        \sum_{j \in J} \sum_{e \in E_j} p_e x_e - \sum_q \lambda_q
        (\sum_{v' \in N(q)} y_{v'} - 1)$。
        \item \textbf{子问题求解}:
        \begin{itemize}
            \item 对于每个列车 $j$,原问题分解为一个最长路径问题(利润最大化)。
            \item 弧的利润根据原始利润 $p_e$ 和由 $\lambda_q$
            导出的惩罚项进行调整。具体来说,如果选择的弧 $e=(u,v)$ 的终点节点 $v_n$ (若为物理节点)
            位于一个或多个 headway 窗口内,其利润会减去相应的 $\sum \lambda_q$。
            \item 使用动态规划算法 在列车各自的时空网络上求解此路径问题。
        \end{itemize}
        \item \textbf{拉格朗日乘子更新}:
        \begin{itemize}
            \item 乘子 $\lambda_q$ 根据次梯度法进行更新: $\lambda_q^{k+1} =
            \max(0, \lambda_q^k + \eta \cdot (\sum_{v' \in N(q)}
            y_{v'}^{k} - 1))$,其中 $\eta$ 是步长 (\texttt{eta}),$\sum
            y_{v'}^{k}$ 是在第 $k$ 次迭代的子问题解中,所有列车共同导致 headway 约束 $q$
            的冲突节点总占用数。
        \end{itemize}
        \item \textbf{启发式方法构造可行解}:
        \begin{itemize}
            \item 由于拉格朗日松弛的解可能不满足被松弛的 headway 约束,因此采用启发式方法构造可行解。
            \item 列车根据其在对偶子问题中的目标值(包含乘子惩罚的路径利润)进行降序排序。
            \item 依次为每列车寻找一条路径,该路径不与已调度列车的路径发生 headway
            冲突。此步骤也通过求解(不含惩罚项的原始利润的)最长路径问题完成,但在构建图时会移除与已选路径冲突的弧(即,如果一个节点
            $v'$ 因选择后会导致与已占用节点冲突,则不考虑指向 $v'$ 的弧)。
        \end{itemize}
        \item \textbf{界更新与终止}:
        \begin{itemize}
            \item 拉格朗日对偶问题的解(所有列车子问题对偶利润之和加上 $\sum \lambda_q (1 -
            \text{violation})$ 项)提供原始问题最优解的一个上界 (UB) (对于最大化问题)。
            \item 原始启发式得到的最好可行解的目标值作为下界 (LB)。
            \item 算法在达到最大迭代次数、UB-LB 间隙足够小或 $\eta$ 过小时终止。
        \end{itemize}
    \end{itemize}

    \subsection{交替方向乘子法 (ADMM)}
    交替方向乘子法 (ADMM) 是一种用于求解分布式优化问题的迭代算法,此算法将原始问题分解为更易处理的子问题,并通过对偶变量协调这些子问题的解。本次作业实现了 ADMM 算法在时空网络模型中的应用,针对列车路径选择和 headway 约束进行优化,算法伪代码如\ref{alg:algorithm}所示,以下为具体实现细节:

    \begin{algorithm}[h]
        \caption{ADMM算法主要流程}
        \begin{algorithmic}[1]
            \Require 问题数据 (时空网络, 列车信息, headway 定义)
            \Ensure 近似最优且可行的列车路径集合 $X_{best}^*$
            \State 初始化局部占用 $z_{local_{jv}}$, 全局占用 $z_{global_{jv}}$, 对偶变量 $\mu_{jv}$
            \State 初始化惩罚参数 $\rho > 0$
            \State $X_{best}^* \gets \emptyset$,
            $\text{best\_feasibility} \gets \infty$,
            $\text{best\_objective} \gets -\infty$
            \For{$k = 1 \dots K_{max}$ (最大迭代次数)}
                \State \Comment{1. x-update (更新列车路径 $X^{k+1}$ 和局部占用 $z_{local}^{k+1}$)}
                \ForAll{列车 $j \in J$}
                    \State 根据 $p_e, \mu^k, z_{global}^k, \rho$ 计算修正的弧利润
                    \State 求解列车 $j$ 的最长路径问题,得到 $X_j^{k+1}$ 和对应的 $z_{local,j}^{k+1}$
                \EndFor
                \State 合并得到 $X^{k+1}$ 和 $z_{local}^{k+1}$
                \State \Comment{2. z-update (更新全局占用 $z_{global}^{k+1}$)}
                \State 对于每个 (车站, 类型) 对:
                \State \quad 根据 $\mu^k, z_{local}^{k+1}, \rho$ 计算节点占用成本
                \State \quad 使用动态规划求解,选择一组满足 headway 约束的 $(v_t, j^*_{v_t})$
                来确定 $z_{global}^{k+1}$
                \State \Comment{3. $\mu$-update (更新对偶变量 $\mu^{k+1}$)}
                \State $\mu_{jv}^{k+1} \gets \mu_{jv}^k + \rho
                (z_{local,jv}^{k+1} - z_{global,jv}^{k+1})$
                \State (可选) 更新惩罚参数 $\rho$
                \State 评估当前解 $X^{k+1}$ 的目标值和可行性
                \If{当前解优于 $X_{best}^*$}
                    \State 更新 $X_{best}^*$, $\text{best\_feasibility}$,
                    $\text{best\_objective}$
                \EndIf
                \If{满足终止条件 (如残差足够小或达到最大迭代次数)}
                    \State \textbf{break}
                \EndIf
            \EndFor
            \State \Return $X_{best}^*$
        \end{algorithmic}\label{alg:algorithm}
    \end{algorithm}

    \begin{itemize}
        \item \textbf{问题重述}:
        \begin{itemize}
            \item 引入辅助全局变量 $z_{global_{jv}}$,表示列车 $j$ 是否占用节点 $v$。
            \item 目标是最大化(或最小化,取决于 $p_e$ 的定义) $\sum p_e x_e$。
            \item 主要约束是 $z_{local_{jv}} = z_{global_{jv}}$,其中
            $z_{local_{jv}}$ 是由 $x_e$ 导出的局部占用变量 (即如果列车 $j$ 的路径包含
            $v$,则 $z_{local_{jv}}=1$)。
            \item 将 Headway 约束应用于 $z_{global_{jv}}$。
        \end{itemize}
        \item \textbf{ADMM 迭代步骤}:
        \begin{enumerate}
            \item \textbf{x-update}:
            \begin{itemize}
                \item 对于给定的 $z_{global}$ 和对偶变量 $\mu$ (对应
                $z_{local} - z_{global} = 0$ 的拉格朗日乘子),求解关于
                $x_e$ (以及隐式的 $z_{local}$) 的子问题。
                \item 目标函数为 $\sum p_e x_e + \sum_{j,v} \mu_{jv}
                (z_{local_{jv}} - z_{global_{jv}}) +
                \frac{\rho}{2} \sum_{j,v} ||z_{local_{jv}} -
                z_{global_{jv}}||_2^2$。在代码实现中,针对每个列车
                $j$,求解一个最长路径问题。弧 $(u,v')$ 的有效利润被调整为 $p_e -
                (\mu_{j,v'} + \frac{\rho}{2} (1 - 2
                z_{global_{j,v'}}))$,如果 $v'$ 是物理节点。
                \item 使用动态规划求解。
                \item 此步骤得到新的 $x_e$ 和 $z_{local}$。
            \end{itemize}
            \item \textbf{z-update}:
            \begin{itemize}
                \item 对于给定的 $z_{local}$ 和 $\mu$,求解关于 $z_{global}$ 的子问题。
                \item 目标是最小化 $\sum_{j,v} [-\mu_{jv}
                z_{global_{jv}} + \frac{\rho}{2}
                (z_{local_{jv}} - z_{global_{jv}})^2]$,同时满足
                headway 约束 $\sum_{j} z_{global_{jv'}} \le 1$
                对于任意冲突集合 $N(v)$ 中的 $v'$。
                \item 此子问题进一步分解:对于每个 (station, type)
                对,通过动态规划解决。首先计算每个 $(j,v)$ 对的成本 $d_{jv} =
                -\mu_{jv} + \frac{\rho}{2} (1 - 2
                z_{local_{jv}})$。然后对每个 $(s, type)$,找到一组时间点
                $\{t_k\}$ 和对应的列车 $\{j_k^*\}$,使得
                $z_{global_{j_k^*, (s,type,t_k)}}=1$ 且满足
                headway,同时最小化 $\sum d_{j_k^*, (s,type,t_k)}$。
            \end{itemize}
            \item \textbf{$\mu$-update}:
            \begin{itemize}
                \item 更新对偶变量:$\mu_{jv}^{k+1} = \mu_{jv}^k + \rho
                (z_{local_{jv}}^k - z_{global_{jv}}^k)$。
            \end{itemize}
        \end{enumerate}
        \item \textbf{收敛与可行性}:
        \begin{itemize}
            \item 算法迭代直至原始残差 ($||z_{local} - z_{global}||_2$) 足够小且解可行,或达到最大迭代次数。
            \item 最终解的可行性通过检查 $z_{global}$ 是否满足所有列车的路径约束和 headway 约束来验证。
            \item 惩罚参数 $\rho$ 在迭代过程中动态调整 (本次实现中 $\rho$ 乘以1.01进行递增)。
        \end{itemize}
    \end{itemize}


\end{document}